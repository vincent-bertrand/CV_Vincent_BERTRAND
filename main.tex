% !TeX spellcheck = en_GB
% !TeX program = pdflatex
%
% LuxSleek-CV 1.1 LaTeX template
% Author: Andreï V. Kostyrka, University of Luxembourg

\documentclass[11pt, a4paper]{article} 

\usepackage[T1]{fontenc}
\usepackage[utf8]{inputenc}
\usepackage[british]{babel}  
\usepackage[left = 0mm, right = 0mm, top = 0mm, bottom = 0mm]{geometry}
\usepackage[stretch = 25, shrink = 25, tracking=true, letterspace=30]{microtype}  
\usepackage{graphicx}
\usepackage{xcolor}
\usepackage{marvosym}
\usepackage{xurl}
\usepackage[hidelinks]{hyperref}
\usepackage{fontawesome5}

% === Ajouts pour l'alignement des icônes ===
\usepackage{array}                 % colonnes m{..}
\usepackage[export]{adjustbox}     % valign=c pour \includegraphics

\usepackage{enumitem}
\setlist{parsep = 0pt, topsep = 0pt, partopsep = 1pt, itemsep = 1pt, leftmargin = 6mm}

\usepackage{FiraSans}
\renewcommand{\familydefault}{\sfdefault}

\definecolor{cvblue}{HTML}{F7E7CE}

%%%%%%% USER COMMAND DEFINITIONS %%%%%%%%%%%%%%%%%%%%%%%%%%%
\newcommand{\dates}[1]{\hfill\mbox{\textbf{#1}}}
\newcommand{\is}{\par\vskip.5ex plus .4ex}
\newcommand{\smaller}[1]{{\small$\diamond$\ #1}}
\newcommand{\headleft}[1]{\vspace*{3ex}\textsc{\textbf{#1}}\par%
    \vspace*{-1.5ex}\hrulefill\par\vspace*{0.7ex}}
\newcommand{\headright}[1]{\vspace*{2.5ex}\textsc{\Large\color{cvblue}#1}\par%
     \vspace*{-2ex}{\color{cvblue}\hrulefill}\par}
% Helpers d'alignement icônes
\newcommand{\icon}[1]{\raisebox{-0.12\height}{#1}}               % pour FontAwesome
\newcommand{\imgicon}[1]{\includegraphics[height=1.05em,valign=c]{#1}} % si tu veux des PNG/SVG
%%%%%%%%%%%%%%%%%%%%%%%%%%%%%%%%%%%%%%%%%%%%%%%%%%%%%%%%%%%%
% PREAMBULE
\usepackage{xcolor}
\definecolor{cvdark}{HTML}{F7E7CE} % ou \definecolor{cvdark}{gray}{0.15}
\setlength{\fboxsep}{12pt} % "padding" interne

% Macro de section "barre"
\newcommand{\cvsection}[1]{%
  \vspace{0.8em}%
  \noindent
  \colorbox{cvdark}{%
    \parbox{\dimexpr\linewidth-2\fboxsep\relax}{%
      \vspace*{6pt}\textcolor{black}{\bfseries\large #1}\vspace*{6pt}%
    }%
  }%
  \par\vspace{0.6em}%
}


\begin{document}

% Style definitions
\setlength{\topskip}{0pt}
\setlength{\parindent}{0pt}
\setlength{\parskip}{0pt}
\setlength{\fboxsep}{0pt}
\pagestyle{empty}
\raggedbottom

\begin{minipage}[t]{0.33\textwidth} %% Left column -- outer definition
\colorbox{cvblue}{\begin{minipage}[t][5mm][t]{\textwidth}\null\hfill\null\end{minipage}}

\vspace{-.1ex}
\colorbox{cvblue!90}{\color{black}
\kern0.09\textwidth\relax
\begin{minipage}[t][293mm][t]{0.82\textwidth}
\raggedright
\vspace*{1.5ex}

\Large Vincent \textbf{\textsc{Bertrand}} \normalsize 

\null\hfill\includegraphics[width=0.85\textwidth]{photo_VB.jpg}\hfill\null

\vspace*{0.5ex}

\headleft{Résumé}
À la recherche d’un stage dans les domaines de l’IT, je souhaite intégrer une entreprise pour un \textbf{stage de 16 semaines} à partir du 02 mars 2026. Je voudrais mettre à profit mes \textbf{compétences} dans le domaine des \textbf{réseaux et de l'IT} en général.

\headleft{Contact (liens cliquables)}
\small
\renewcommand{\arraystretch}{1.1}
\begin{tabular}{@{}>{\centering\arraybackslash}m{1.6em} l@{}}
\icon{\faIcon{map-marked-alt}} &
Nancy (54), Bar-Sur-Aube (10) \\
\icon{\faEnvelope} & \href{mailto:XXXXXXX@gmail.com}{vincent.bertrand10p@gmail.com} \\
\icon{\faPhone}    & \href{tel:+910000000000}{06\,52\,23\,54\,27} \\
\icon{\faLinkedin} & \href{https://www.linkedin.com/in/vincent-bertrand-5aa762177}{Mon profil LinkedIn} \\
\href{https://www.linkedin.com/in/vincent-bertrand-5aa762177}{%
  \shortstack[l]{\nolinkurl{linkedin.com/in/}\\\nolinkurl{vincent-bertrand-5aa762177}}%
} \\
\icon{\faGithub}   & \href{https://github.com/vincent-bertrand}{github.com/vincent-bertrand} \\
\end{tabular}
\normalsize

% --- Si tu préfères tes images LinkedIn/GitHub, remplace la ligne correspondante par :
% \imgicon{icons8-linkedin-48.png} & \href{https://www.linkedin.com/in/vincent-bertrand-5aa762177}{Mon profil LinkedIn} \\
% \imgicon{icons8-github-30.png}   & \href{https://github.com/vincent-bertrand}{github.com/vincent-bertrand} \\

\headleft{Langues}
Français: \textbf{Maternelle} \\[0.5ex]
Anglais: \textbf{B2 (TOEIC : 645/990)} \\[0.5ex]
Espagnol: \textbf{A1 (débutant)}

\headleft{Compétences}
\begin{itemize}
    \item \textbf{Réseaux} (Déploiement et sécurisation d’un SI).
    \item \textbf{Langages :}
    \begin{itemize}
        \item Présentation : LaTex, HTML, CSS.
        \item Programmation : C, Python, Java, SQL, JS, VBA (Excel).
    \end{itemize}
    \item \textbf{IoT} : ESP32, MQTT, NodeRed, RESTful API.
    \item \textbf{Méthodes numériques} et \textbf{traitement du signal} (Matlab, FFT).
    \item \textbf{Veille technologique}.
    \item \textbf{Gestion de projet et travail collaboratif} : GitHub (versionnage), Gantt.
\end{itemize}

\end{minipage}%
\kern0.09\textwidth\relax
}
\end{minipage}%
\hskip2.5em
\begin{minipage}[t]{0.56\textwidth}
\setlength{\parskip}{0.8ex}

\vspace{2ex}

\cvsection{Experience}

\textsc{ASSOCIÉ GÉRANT} à \textit{EARL Champagne Hervé BERTRAND, Voigny.}  \dates{Juin 2020-ACTUEL} \\
\is
\textbf{Compétences : anticipation, gestion des incidents, autonomie}. \\
\smaller{Travaux de la vigne, palissage, vendanges.
Travaux en cave, approvisionnement, logistique, dégorgement, embouteillage, préparation d’expéditions et accueil de clients.} \\
\textbf{Gestion d’entreprise}. 
\vspace{1cm} \\
\textsc{Conducteur de travaux} à \textit{Axians (Santerne Est Telecoms SAS), Ay-sur-Moselle, Grand Est, France},  \dates{Avril 2025 - Juin 2025} \\
\is
\textbf{Compétences : Préparation de devis, études techniques, visites sur site, réaménagement, respect des délais, gestion de projets, veille technologique}. \\
\smaller{Pendant ce stage, j’ai contribué à la gestion et à l’optimisation d’infrastructures TowerCo (pylônes télécoms et équipements radio). Réalisé au sein d'Axians, société spécialisée dans l'ICT (Information \& Communication Technologie) filière de Vinci Energies, elle-même au sein du groupe Vinci.} \\
\href{https://github.com/vincent-bertrand/portfolio/blob/8eba2640d10e6c2511ef1cbfb93acad47c537a24/2A_BERTRAND_Axians.pdf}{\textbf{Compte rendu du stage 2ème année BUT R\&T (lien vers GitHub)}}
\is

\cvsection{FORMATION}

\textsc{BUT R\&T PARCOURS ROM}, \textit{IUT Nancy-Brabois, Villiers-lès-Nancy}. \dates{2024 - 2026} \\

\is
\textsc{LICENCE SCIENCES POUR L’INGÉNIEUR (parcours Systèmes Numériques, Productique et Réseaux}, \textit{FST Nancy, Université de Lorraine}.\dates{2019 - 2024} \\

\is
\textsc{BACCALAURÉAT STI2D OPTION (EE).} \textit{Lycée Henri Loritz, Nancy},\dates{2017 - 2019} \\

\cvsection{RÉALISATION ET CONCOURS}
\smaller{\textsc{\textbf{CONCOURS : “MON PROJET EN 5 MINUTES”}} \textit{club eea 9ème édition, émerainville}} \dates{JUIN 2024} \\
\textit{Lors du concours \textbf{“Mon Projet en 5 Minutes”}, avec mon équipe Nous sommes allés jusqu'à la \textbf{3e place} en finale (le 11 juin 2024 lors de la 9 ème édition) avec \textbf{Poub’Ul}, un \textbf{projet IoT} visant à optimiser la collecte des déchets à l’Université de Lorraine. J’ai assuré la gestion de la partie réseau et de la transmission des données vers le serveur, alliant \textbf{rigueur} et \textbf{esprit d’analyse} pour une solution efficace et connectée.}

\cvsection{CENTRES D’INTÉRÊT}

\textit{Sciences (sciences de l’information; viticulture, œnologie) - Sport (trail) -
Automobile -
Aéronautique}
\end{minipage}

\end{document}
